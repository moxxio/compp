\section{Mathematik}

\subsection{Kombinatorik}

\subsubsection{Stirlingzahlen zweiter Art}
Die Stirlingzahlen zweiter Art $S_{n, k}$ oder ${n \brace k}$ bezeichnet die Anzahl von $k$-Partitionen einer $n$-elementigen Menge. Es gilt
\begin{equation*}
	S_{n, k} = \frac{1}{k!} \sum_{i=0}^{k} (-1)^i {k \choose i} (k-i)^n \quad \text{und} \quad
	S_{n, k} = \begin{cases}
	0, &n < k, \\
	0, &n \in \mathbb{N}, k = 0, \\
	0, &n = 0, k \in \mathbb{N}, \\
	1, &n = k = 0, \\
	S_{n-1, k-1} + kS_{n-1,k}, &\text{sonst}.
	\end{cases}
\end{equation*}

\subsection{Graphentheorie}
